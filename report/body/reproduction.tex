\section{Typilus模型复现}
\label{sec:reproduction}

% 本章记录Typilus模型在NaturalCC框架下的复现过程,
% 包括复现实验设置、训练过程与结果,以及复现过程中遇到的主要技术问题和对应的解决办法。

本章记录 Typilus 模型在 NaturalCC 框架下的复现过程,
包括复现实验的基本配置、实现细节,以及过程中遇到的主要问题和解决方法。
% 基于 NaturalCC 框架实现基于 Transformer 的类型推断,
% 在后文的对比实验中,将以本章复现得到的 Typilus 结果,
% 与基于 Transformer 的序列模型进行简单比较。

% \subsection{复现实验设置}

% 完成环境配置和数据准备后,使用 NaturalCC 仓库中内置的 Typilus 任务配置作为起点。
% 为避免引入额外变量,复现实验尽量沿用原实现的设置,仅在显存等资源限制下对 batch size 等少数参数做小幅调整。

% Typilus 模型的主要超参数包括嵌入维度、GGNN 层数、隐藏维度、学习率和批量大小等。
% 本实验采用 NaturalCC 提供的默认配置:嵌入维度 512、GGNN 层数 8、triplet margin 1.0,优化器和学习率调度保持默认。
% 训练、验证和测试均基于前文构建的 Typilus 数据集划分。

% 训练过程中在验证集上监控 Top-$k$ 准确率和 MRR 指标,并保留若干性能较好的检查点,用于后续对比和误差分析。

\subsection{复现实验设置}

完成环境配置和数据准备后,复现实验使用 NaturalCC 仓库中内置的 Typilus 任务配置作为起点。
为避免引入额外变量,主要沿用原实现的设置,仅在显存等资源限制下对 batch size 等少数参数做了适当下调。

Typilus 模型的主要超参数包括嵌入维度、GGNN 层数、隐藏维度、学习率和批量大小等。
本实验采用 NaturalCC 提供的默认配置:嵌入维度 512、GGNN 层数 8、triplet margin 1.0,
优化器与学习率调度保持默认,训练、验证和测试均基于前文构建的 Typilus 数据集划分进行。
此外对基于 Transformer 的类型推断模型进行了尝试。

% 此外实现了基于 Transformer 的类型推断模型,
% 使用同一 Typilus 数据集,但将代码表示从程序图转换为 token 序列,
% 基于 NaturalCC 中的代码实现。

% 训练过程中在验证集上监控 Top-$k$ 准确率和 MRR 等指标,
% 并保留若干性能较好的检查点,作为后续误差分析和对比实验的参考。


% \subsection{训练过程与结果}

% 在解决环境与数据预处理相关问题之后,Typilus模型可以在GPU上正常训练。
% 从损失变化来看,前几个epoch中训练损失较快下降,
% 随后在相对稳定的区间内震荡,验证集Top-1和Top-$k$准确率随之提升并逐渐趋于平稳。

% 受数据规模、预处理细节以及实现差异等因素影响,
% 本实验复现的绝对指标与原论文报告存在一定差距,
% 但整体趋势与预期一致:相较于随机预测,Typilus在验证集和测试集上显著优于简单基线,
% 能够利用程序图结构和上下文信息,对常见类型给出较为可靠的预测。
% 在后续章节中,Typilus复现模型作为图神经网络基线,与基于Transformer的序列模型进行对比分析。

% \subsection{关键技术问题与解决}

% 复现过程中遇到的主要技术问题集中在图数据格式、显存占用以及词典和特殊token的处理等方面。

\subsection{技术问题与解决}

% 复现 Typilus 并在此基础上加入 Transformer 序列基线的过程中,主要遇到几类问题:
复现 Typilus 的过程中主要遇到几类问题:
图数据字段不一致、距离计算带来的显存压力,以及在构造 Transformer 训练数据时
代码序列与类型标签的对齐和特殊 token 处理问题。
本节按类别对这些问题及其解决方法作简要说明。

\subsubsection{图数据与节点特征不一致}

在最初的 Typilus 训练尝试中,模型在前向传播阶段出现 \texttt{KeyError: 'subtoken'} 错误,
提示 DGL 图的节点特征中缺少 \texttt{subtoken} 字段。
排查发现,NaturalCC 某些版本对节点特征字段名有所调整,
而预处理脚本仍按照旧字段名输出,导致模型在访问节点特征时找不到对应键。

为解决这一问题,在数据预处理和模型入口两侧统一了字段命名:
一方面更新预处理脚本,保证图数据中包含模型期望的 \texttt{subtoken} 或等价字段;
另一方面检查 NaturalCC 中 Typilus 模型的输入接口,避免重复删除或覆盖同一字段。
修正后,图数据能够顺利加载,编码器可以正常读取节点的 token 表示。

\subsubsection{距离计算与显存占用}

在训练和验证阶段,Typilus 需要根据节点表示和类型标签计算相似度,
用于 triplet 损失和 Top-$k$ 指标的统计。
原始实现中,数据集在 \texttt{collate} 阶段为 batch 内所有目标节点构造了一个
大小为 $B \times B$ 的稠密邻接矩阵,其中 $B$ 为该 batch 中参与监督的节点个数,
邻接矩阵元素指示两节点类型是否相同。
在此基础上,triplet 损失一次性构造所有节点对的距离矩阵(同为 $B \times B$),
当 $B$ 较大时会占用大量显存,实际运行中触发 CUDA OOM 错误。

为降低显存开销,一方面在数据集构建阶段不再显式构造稠密邻接矩阵,
而是仅传递长度为 $B$ 的类型标签向量;
另一方面在 \texttt{TripletCriterion} 中实现了基于标签的“内存友好”计算路径:
当输入为一维标签向量时,将 batch 内的节点表示按固定大小划分为若干小块,记每块大小为 $K$。
对于每一块,将该块中的节点表示作为 anchor,与整个 batch 的表示计算
一个 $K \times B$ 的距离矩阵,根据标签现场生成正负样本掩码,并计算对应的 triplet 损失分量,
再在各个块之间累加得到完整损失。
这一改动不改变 triplet 损失的定义和数值形式,
但将原本一次性构造的 $B \times B$ 距离矩阵拆分为多次 $K \times B$ 的小矩阵计算,
显著降低了前向和验证阶段的显存峰值,使 Typilus 模型能够在单卡显存有限的条件下稳定完成训练和评估。

\subsubsection{Transformer 序列数据与标签对齐}

在构造基于 Transformer 的序列模型时,需要将 Typilus 的图数据转换为
一行一条样本的 \texttt{.code}/\texttt{.type} 文本格式。
最初直接使用 \texttt{nodes} 数组或随意添加起止标记,加载数据时频繁触发
\texttt{len(tokens) != len(labels)} 的断言。

问题主要在于:一方面 \texttt{token-sequence} 与 \texttt{nodes} 的长度并不总是一致,
真正的代码序列应以 \texttt{token-sequence} 中的 id 为准;
另一方面个别样本在生成类型标签时越界或为空,写入文本后再用 \texttt{split()} 解析时,
token 内部的空白字符还会导致长度进一步错位。

为保证严格对齐,转换过程中采用 \texttt{token-sequence} + 词典反查得到代码 token,
类型标签由 \texttt{supernodes} 注解生成,并要求每条样本满足
\texttt{len(token\_ids) == len(type\_labels)},否则直接丢弃;
写文件前对 token 做简单清洗,去除或替换内部空白,避免读取阶段被拆分成多个“伪 token”。

\subsubsection{CUDA 索引错误与特殊 token 处理}

在 Transformer 训练阶段多次出现 \texttt{CUDA error: device-side assert triggered},
定位到 embedding 查表时输入了非法索引。
排查发现并非 “id 超出词表大小”,而是存在大量 \texttt{-1} 被当作 token id 使用。

原因与特殊 token 初始化有关:Dictionary 在某些加载路径下没有正确添加
\texttt{<pad>}、\texttt{<unk>}、\texttt{<s>}、\texttt{</s>},
导致 \texttt{pad\_idx} 和 \texttt{unk\_idx} 仍为 \texttt{-1},
OOV token 和 padding 最终都被映射为 \texttt{-1},从而触发设备端断言。

为避免这一问题,在构造和加载词典时统一采用 Dictionary 的默认初始化,
由其自动创建上述四个特殊符号,并在数据转换脚本中只使用这一套约定,
对 Typilus 中的 \texttt{[PAD]}、\texttt{[UNK]} 等写法显式映射到对应符号。
同时增加简单的覆盖检查,确认 \texttt{.code}/\texttt{.type} 中的 token 均能在词典中找到,
调整后 CUDA 索引错误不再出现,序列模型可以在 Typilus 数据上稳定训练。```

% \subsubsection{其他技术细节}

% 复现过程中还遇到若干次要问题,例如:
% \begin{itemize}
%     \item 不同版本 PyTorch 在 checkpoint 序列化格式上的差异,需要通过指定 \texttt{strict = False} 等方式兼容加载;
%     \item DGL 批处理接口在版本更新后行为略有变化,需要根据最新文档调整图合并与拆分方式;
%     \item 部分算子在低精度下数值不稳定,因此训练中统一采用单精度浮点(float32)。
% \end{itemize}
% 这些问题通过查阅官方文档、阅读错误堆栈并逐步修改代码得到解决。

\subsection{复现小结}

在前述环境配置、数据准备以及关键问题修正之后,Typilus 模型和 Transformer 模型可以在 NaturalCC 框架下正常训练和验证。
其中,Typilus 模型能够在给定数据集上收敛;Transformer 模型受限于时间问题,仅跑通了训练流程。