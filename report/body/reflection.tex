\section{课程学习感想}
\label{sec:reflection}

通过《图神经网络导论》课程的学习和本次实验,
我在理论理解、实践能力等方面均有收获。

理论知识方面,课程系统地介绍了图神经网络的基本概念、核心方法和典型应用。
谱域方法和空间域方法的对比让我认识到同一问题可以有不同的形式化方式,
不同视角各有优劣。
在Typilus复现过程中,我体会到消息传递机制的实际效果:
通过图结构传递信息,模型能够自动学习程序元素之间的依赖关系,
这种端到端的学习范式相比传统方法具有更好的适应性。

理论与实践之间存在一定gap。
实验中遇到的许多工程问题(如显存管理、数据格式兼容、批处理策略等)
需要结合理论知识和实际调试才能解决。
这个过程让我认识到,扎实的理论基础是解决实际问题的重要前提。

实践能力方面,
本次实验涉及环境配置、数据处理、模型训练和结果分析等多个环节。
在数据处理阶段,遇到了字符编码、格式兼容、特殊token处理等问题,
通过系统性调试逐步解决,提升了调试能力。
在模型训练阶段,显存溢出、训练不稳定等问题让我认识到
理论设计与工程实现之间的差距,
熟悉了监控训练过程、调整超参数、优化显存占用。
此外,通过阅读Typilus论文和NaturalCC源码,
提升了文献阅读和代码理解能力,
从核心思想出发逐步深入细节。



通过《图神经网络导论》课程的学习和本次实验,
我不仅掌握了图神经网络的理论知识,
更重要的是积累了深度学习项目的实践经验。
图神经网络在代码智能、生物信息等领域展现出巨大潜力。
本次实验让我对这一领域有了深入理解,
也激发了进一步探索的兴趣。
在未来的学习中,我将继续关注该领域的最新进展,
尝试将其应用于实际问题。
