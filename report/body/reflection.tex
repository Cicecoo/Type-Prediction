\section{课程学习感想}
\label{sec:reflection}

通过《图神经网络导论》课程的学习和本次实验,
我在理论理解、实践能力和科研思维等方面均有收获。

\subsection{理论知识的收获}

课程系统地介绍了图神经网络的基本概念、核心方法和典型应用。
谱域方法和空间域方法的对比让我认识到同一问题可以有不同的形式化方式,
不同视角各有优劣。

在Typilus复现过程中,我体会到消息传递机制的实际效果:
通过图结构传递信息,模型能够自动学习程序元素之间的依赖关系,
这种端到端的学习范式相比传统方法具有更好的适应性。

理论与实践之间存在一定gap。
实验中遇到的许多工程问题(如显存管理、数据格式兼容、批处理策略等)
需要结合理论知识和实际调试才能解决。
这个过程让我认识到,扎实的理论基础是解决实际问题的重要前提。

\subsection{实践能力的提升}

本次实验是一次完整的深度学习项目实践,
涉及环境配置、数据处理、模型训练和结果分析等多个环节。

在环境配置阶段,我学会了处理版本依赖冲突,
掌握了深度学习环境的管理方法。

在数据处理阶段,遇到了字符编码、格式兼容、特殊token处理等问题,
通过系统性调试逐步解决,提升了调试能力。

在模型训练阶段,显存溢出、训练不稳定等问题让我认识到
理论设计与工程实现之间的差距,
学会了监控训练过程、调整超参数、优化显存占用。

实验设计方面,我认识到科学实验需要严谨的设计:
明确目标、控制变量、设置对比、记录过程。
养成了记录实验日志和版本控制的习惯,
确保工作的可复现性。

此外,通过阅读Typilus论文和NaturalCC源码,
提升了文献阅读和代码理解能力,
学会了从核心思想出发逐步深入细节。

\subsection{科研思维的培养}

实验过程中,我逐渐培养了批判性思考的习惯。
在分析模型性能时,不仅关注准确率数字,
还思考背后的原因:类型分布是否平衡?图规模是否影响?
这种思考方式帮助我更深入地理解问题本质。

同时认识到系统性方法的重要性。
解决问题时,先复现、提出假设、设计验证、记录过程,
这种系统化的方法比盲目尝试更高效。

虽然实验主要独立完成,
但在遇到困难时也寻求了同学和社区的帮助。
这让我认识到科研需要交流协作,
学会提出清晰的问题并有效沟通是重要的能力。

\subsection{总结}

通过《图神经网络导论》课程的学习和本次实验,
我不仅掌握了图神经网络的理论知识,
更重要的是积累了深度学习项目的实践经验,
培养了系统性的科研思维。

图神经网络在代码智能、生物信息等领域展现出巨大潜力。
本次实验让我对这一领域有了深入理解,
也激发了进一步探索的兴趣。

在未来的学习中,我将继续关注该领域的最新进展,
尝试将其应用于更多实际问题。

感谢老师的指导和同学们的帮助。
