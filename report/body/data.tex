\section{实验环境与数据准备}
\label{sec:data}

本章介绍实验使用的硬件和软件环境,以及 Typilus 数据集在 NaturalCC 框架中的构建过程。

\subsection{实验环境}

实验在一台运行 Linux 的 NVIDIA GPU 服务器上进行。
通过 Conda 创建独立环境 \texttt{naturalcc},主要依赖包括 Python、PyTorch、DGL(Deep Graph Library)
% 和 Transformers 
等深度学习与图学习库。

环境配置中遇到的关键问题主要与库版本有关:
% \begin{itemize}

     \textbf{PyTorch、DGL 与 CUDA 的兼容性}:
    服务器上使用的是 PyTorch 2.4 和 CUDA 12.4,NaturalCC 默认的 DGL 版本无法正常导入,
    需要根据官方说明安装与该版本 PyTorch 对应的 DGL 预编译包,才能解决动态链接库缺失的问题。
    
    \textbf{系统库与 Python 库的冲突}:
    运行过程中出现 \texttt{libp11-kit.so.0} 依赖 \texttt{libffi} 符号缺失的错误,
    通过在 Conda 环境中安装合适版本的 \texttt{p11-kit}、\texttt{gnutls} 和 \texttt{libffi} 进行修复;
    对于 \texttt{ruamel.yaml} 移除 \texttt{safe\_load} 接口带来的不兼容,
    通过修改 NaturalCC 内部的 YAML 读取函数解决。
    % 此外,还根据文档要求固定了 \texttt{transformers} 和 \texttt{tokenizers} 等库的版本。
% \end{itemize}

NaturalCC 使用 \texttt{master} 分支与实验文档保持一致,
并通过环境变量 \texttt{NCC} 指定 Typilus 数据根目录,使数据处理与训练脚本可以在固定路径下读取和保存数据。

\subsection{数据集准备}

\subsubsection{数据来源与原始流程}

Typilus 使用从 GitHub 收集的大规模开源 Python 项目作为训练数据。
原始流程在 Docker 容器中运行官方脚本,依次完成仓库克隆、调用 \texttt{pytype} 做静态类型分析,
并生成代码属性图及 \texttt{dataset.spec} 等元数据。

在实验中,选择绕过Docker,将这套流程改写为本地脚本,方便调试和恢复。
根据官方脚本编写了 \texttt{prepare\_data\_local.sh},在本地完成仓库克隆、(可选的)近重复检测、
\texttt{pytype} 分析以及数据集划分,不再依赖Docker容器。
这样可以直接在服务器上查看日志和中间结果,长时间任务中断后也便于继续运行。

\subsubsection{图数据与 NaturalCC 格式转换}

在获得 Typilus 原始图数据后,需要将其转换为 NaturalCC 支持的统一格式。
先调用 \texttt{ncc\_dataset.prepare\_dataset('typilus', \dots)} 读取 Typilus 图文件和 \texttt{dataset.spec},
    生成按照训练集、验证集和测试集划分的属性数据集(attributes),其中包含节点特征、边关系和类型标签等信息;
    然后根据官方提供的 \texttt{typilus.yml} 配置运行预处理脚本,将属性数据集转换为内存映射(mmap)格式,
    构建节点、边和类型字典,并生成用于训练的二进制文件。
% \end{itemize}

预处理过程中,一些图样本由于别名替换失败或字符编码异常,出现节点被解析为 \texttt{None} 的情况,
会导致后续脚本报错。
为此在数据处理代码中增加了异常捕获和完整性检查,对不合法样本进行过滤。
% 保证整条数据管线可以在大规模数据上顺利跑完。

% 在对比实验部分,还基于 Typilus 图数据构造了面向 Transformer 序列模型的数据集,
% 将图上的类型预测任务改写为 token 序列上的标注任务,
% 并重新构建词典,统一 \texttt{[PAD]}、\texttt{[UNK]}、\texttt{<s>}、\texttt{</s>} 等特殊符号的命名和映射。
% 相关细节在后续章节中进一步说明。

\subsubsection{数据特点}

Typilus 数据集中类型分布非常不均衡:
\texttt{int}、\texttt{str}、\texttt{bool} 等常见类型的样本远多于长尾类型,
在训练和评估时容易造成模型偏向高频类型。
不同代码文件生成的图规模也差异较大,从十几个节点到数千个节点不等,
在训练时需要配合合适的批处理策略控制显存占用。

\subsection{数据加载与批处理策略}

训练阶段使用 DGL 提供的图数据加载器进行批处理和多进程加载,以提高数据读取效率。
考虑到图规模差异较大,采用基于节点数量的动态批处理策略:
为每个 mini-batch 设置节点总数上限,根据图的大小自动调整批内样本数量,
在保证 GPU 利用率的同时避免显存占用过高或批次之间差异过大。

% 这一数据加载和批处理方案在 Typilus 图模型和后续的 Transformer 序列模型中保持一致,
% 便于在相同数据划分上比较不同模型的效果。
