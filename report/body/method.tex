\section{方法原理与设计}

本章首先结合 Pacman 游戏规则进行问题建模,之后逐一介绍各算法的原理和设计。

\subsection{问题建模}

在将强化学习算法应用于 Pacman 游戏之前,我们首先需要明确游戏的运行机制,并将问题建模为马尔可夫决策过程(Markov Decision Process, MDP)。

\subsubsection{Pacman 游戏环境}

本实验采用的 Pacman 游戏环境基于开源 Pacman 项目改编,并封装为符合 Gymnasium 标准的强化学习环境。

\cfig{game_start}{0.6}{Pacman 游戏画面示例}

游戏的基本要素包括:
地图(Layout)由墙壁(wall)、通道、食物(food)、胶囊(capsule)以及角色初始位置组成,以文本文件形式存储,如图\ref{fig:text}所示。
%,使用字符表示不同元素:\texttt{\%} 表示墙壁,\texttt{.} 表示食物,\texttt{o} 表示胶囊,\texttt{P} 表示吃豆人初始位置,\texttt{G} 表示鬼的初始位置。
吃豆人由智能体控制,可以在迷宫中上下左右移动(North, South, East, West),也可以选择停留(Stop),其位置用坐标 $(x, y)$ 表示。
食物和胶囊分布在地图的通道中,吃豆人吃掉食物可以获得分数,吃掉胶囊则会使所有鬼进入恐慌状态。

% 文本表示示例:
\cfig{text}{0.5}{地图文本表示}

鬼(Ghost)是随机移动的敌对角色,通常采用随机策略选择动作,有正常状态和恐慌状态两种:正常状态下碰到吃豆人会导致游戏失败;
而在恐慌状态下(吃豆人吃掉胶囊后的 40 个时间步内),被吃豆人吃掉后会在初始位置复活并解除恐慌。


游戏的终止条件包括:吃豆人吃掉所有食物则游戏胜利;吃豆人碰到处于正常状态的鬼则游戏失败;以及达到最大步数限制(避免产生过长的 episode 而进行截断,设置为 1000 步)。


\subsubsection{马尔可夫决策过程建模}

强化学习算法的核心是通过与环境交互来学习最优策略,而 MDP 建模通过定义状态、动作、转移概率和奖励函数,将这一交互过程形式化为值函数的优化问题。
MDP 的马尔可夫性质(下一状态仅依赖于当前状态和动作)使得贝尔曼方程成立,从而可以递归地求解最优值函数和策略。
下面将 Pacman 游戏的各个要素形式化为 MDP 的组成部分。

\textbf{状态空间 $\mathcal{S}$}:游戏中的完整信息构成状态。
具体地,状态包括吃豆人的位置 $(x_p, y_p)$ 和移动方向,所有鬼的位置 $(x_g^i, y_g^i)$ 和恐慌计时器 $t_{\text{scared}}^i$($i = 1, 2, \ldots, n_g$),
剩余食物的分布(用布尔矩阵 $F$ 表示,$F[x][y] = 1$ 表示位置 $(x,y)$ 有食物),剩余胶囊的位置集合 $C$,以及当前累积分数 $s$。
这些信息描述了游戏的当前局面,并满足马尔可夫性质:给定当前状态和动作,下一状态的分布与历史无关。

值得注意的是,状态空间的大小随地图尺寸呈指数级增长。
以 \texttt{smallClassic} 地图(尺寸为 $20 \times 7$)为例,假设有 2 个鬼和 30 个食物点,完整状态空间的规模约为
$|\mathcal{S}| \approx 140 \times 140^2 \times 40^2 \times 2^{30} \approx 10^{17}$,
庞大的状态空间给强化学习算法的设计和实现带来了挑战。

\textbf{动作空间 $\mathcal{A}$}:智能体在每个时间步可以选择五个基本动作之一,即
$\mathcal{A} = \{\text{North}, \text{South}, \text{East}, \text{West}, \text{Stop}\}$。
在选择动作时还需考虑墙壁约束,只有合法动作(不会撞墙的动作)可被选择。

\textbf{状态转移函数 $P(s'|s,a)$}:游戏规则决定了状态如何转移,这部分由开源 Pacman 项目代码实现。
给定当前状态 $s$ 和吃豆人的动作 $a$,状态转移包括确定性和随机性:
吃豆人根据动作 $a$ 移动到新位置、食物和胶囊在被吃掉时更新、恐慌计时器的递减等,是确定性的;
而鬼根据其随机策略选择动作并移动,引入了转移的随机性。
因此,虽然吃豆人的行为是确定的,但由于鬼的随机移动,整体状态转移 $P(s'|s,a)$ 是一个概率分布。

\textbf{奖励函数 $R(s,a,s')$}:游戏的计分规则被形式化为奖励函数,以引导智能体的学习。
具体地,吃掉一个食物获得 $+10$ 分,
吃掉一个恐慌状态的鬼获得 $+200$ 分,
吃掉所有食物(胜利)额外获得 $+500$ 分,
碰到正常状态的鬼(失败)扣除 $500$ 分,
每走一步扣除 $1$ 分作为时间惩罚。
吃掉胶囊本身不直接得分,但会使鬼进入恐慌状态,间接影响后续奖励。

\textbf{折扣因子 $\gamma$}:设置 $\gamma$ 用于平衡即时奖励和长期奖励。
例如,较小的折扣因子可能使智能体更重视短期奖励,而不是过度探索。

通过以上建模,Pacman 游戏被形式化为一个 MDP 五元组 $(\mathcal{S}, \mathcal{A}, P, R, \gamma)$。
智能体的目标是学习最优策略 $\pi^*$,最大化从初始状态开始的期望累积折扣奖励:
\begin{equation}
    \label{eq:optimal_policy}
    \pi^* = \arg\max_\pi \mathbb{E}_\pi \left[ \sum_{t=0}^\infty \gamma^t R_t \right]
\end{equation}


% Method 1

\subsection{Monte Carlo Learning 原理与设计}

\subsubsection{基本原理}

MC learning 使用采样得到的 episode 数据来估计值函数。
与需要环境模型的动态规划方法不同,MC learning 仅需与环境交互即可学习。

动作值函数 $Q(s,a)$ 定义为从状态 $s$ 执行动作 $a$ 后,遵循策略 $\pi$ 所能获得的期望累积折扣奖励:
\begin{equation}
    \label{eq:q_function}
    Q^\pi(s,a) = \mathbb{E}_\pi \left[ \sum_{t=0}^\infty \gamma^t R_{t+1} \mid S_0 = s, A_0 = a \right]
\end{equation}

% MC learning 通过实际经验来估计这一期望值。
由于环境模型位置,期望无法直接计算。MC learning 的基本思想是用实际回报 $G_t$ 的样本平均来估计 $Q(s,a)$。
根据大数定律,随着采样 episode 数的增加,样本平均将收敛到真实的期望值。

具体而言,智能体与环境交互产生完整的 episode:
\begin{equation}
    \label{eq:trajectory}
    S_0, A_0, R_1, S_1, A_1, R_2, \ldots, S_T
\end{equation}
其中 $S_T$ 是终止状态。对于 episode 中出现的每个状态-动作对 $(s,a)$,可以计算从该时刻开始的实际回报(return):
\begin{equation}
    \label{eq:return}
    G_t = R_{t+1} + \gamma R_{t+2} + \gamma^2 R_{t+3} + \cdots + \gamma^{T-t-1} R_T
\end{equation}

通过增量式更新的方式,MC learning 可以在每个 episode 结束后,利用实际回报 $G_t$ 来逐步改进对 $Q(s,a)$ 的估计。


在遍历 episode 的每一步时,可能重复遇到某些状态-动作对,相应地在更新值函数时有两种策略:first-visit 和 every-visit。
first-visit 只在 episode 中首次访问 $(s,a)$ 时更新 $Q(s,a)$,而 every-visit 在每次访问时都进行更新。
本实验采用 every-visit 策略,配合增量式更新来提高数据利用率:
\begin{equation}
    \label{eq:mc_update}
    Q(s,a) \leftarrow Q(s,a) + \alpha \left( G_t - Q(s,a) \right)
\end{equation}
其中 $\alpha \in (0,1]$ 是学习率,控制新样本对估计值的影响程度。

此外,为了在探索(exploration)和利用(exploitation)之间平衡,采用 $\epsilon$-Greedy 贪心策略选择动作:
以概率 $1-\epsilon$ 选择当前估计下的最优动作 $\arg\max_a Q(s,a)$,并以概率 $\epsilon$ 随机选择动作(作为一种尝试)。

% 可选:插入 MC 算法流程图
% \cfig{mc_flowchart}{0.6}{蒙特卡洛学习流程}


\subsubsection{算法设计}

基于上述原理,MC learning 的训练过程可以形式化为算法 \ref{alg:mc_learning}。
% 算法采用 every-visit 策略配合增量式更新,通过反复与环境交互来逐步改进 Q 值估计和策略。
算法流程如 \ref{alg:mc_learning} 所示,推理时只需将 $\alpha$ 和 $\epsilon$ 设为 0 即可。

\begin{algorithm}[h]
    \caption{Monte Carlo Learning}
    \small
    \label{alg:mc_learning}
    \begin{algorithmic}[1]
        \Require 环境 $\text{env}$,最大训练 episode 数 $N$
        \Ensure 学习得到的 Q 值函数 $Q$
        
        \State 初始化:$Q(s,a) \leftarrow 0$ 对所有 $s \in \mathcal{S}, a \in \mathcal{A}$
        \State 设置超参数:$\alpha$, $\epsilon$, $\gamma$
        
        \For{$episode = 1$ to $N$}
            \State $s \leftarrow \text{env.reset()}$ \Comment{初始化环境}
            \State $\text{buffer} \leftarrow []$ \Comment{初始化 episode 缓冲区}
            
            \While{episode 未结束}
                \State 以 $\epsilon$-Greedy 策略选择动作:
                \State \quad $a \leftarrow \begin{cases} 
                    \arg\max_{a'} Q(s, a') & \text{概率 } 1-\epsilon \\
                    \text{random}(\mathcal{A}) & \text{概率 } \epsilon
                \end{cases}$
                \State $s', r \leftarrow \text{env.step}(a)$ \Comment{执行动作并观察转移}
                \State $\text{buffer.append}((s, a, r))$ \Comment{存储经验}
                \State $s \leftarrow s'$
            \EndWhile
            
            \State $G \leftarrow 0$ \Comment{从终止状态开始反向计算回报}
            \For{$t = |\text{buffer}| - 1$ down to $0$}
                \State $(s_t, a_t, r_{t+1}) \leftarrow \text{buffer}[t]$
                \State $G \leftarrow r_{t+1} + \gamma G$ \Comment{累积折扣回报}
                \State $Q(s_t, a_t) \leftarrow Q(s_t, a_t) + \alpha (G - Q(s_t, a_t))$ \Comment{更新 Q 值}
            \EndFor
        \EndFor
        
        \Return $Q$
    \end{algorithmic}
\end{algorithm}

每个训练轮次包括两个阶段:数据收集阶段(第 5-12 行)和值函数更新阶段(第 14-18 行)。
在数据收集阶段,智能体使用 $\epsilon$-Greedy 策略与环境交互,生成完整的 episode 并存储在缓冲区中。
通过 $\epsilon$ 控制在利用当前最优动作和随机探索之间的平衡,确保算法能够充分探索状态空间。

episode 结束后进入更新阶段,算法从后向前遍历缓冲区中的经验。
对于每个时间步 $t$,首先根据递推关系 $G_t = R_{t+1} + \gamma G_{t+1}$ 计算该步的累积折扣回报 $G_t$(其中 $G_T = 0$),
然后使用式 \eqref{eq:mc_update} 更新对应状态-动作对的 Q 值。
这一增量式更新方式使得 Q 值逐渐向真实值函数靠拢,同时学习率 $\alpha$ 控制了新样本对估计的影响程度。

对于 Q 函数,这里实现为字典结构,以完整游戏状态为键,存储对应的动作值。
状态与 MDP 建模部分定义相同,
由于 Pacman 游戏的状态空间可能很大,字典仅存储实际访问过的状态-动作对,未访问的对其 Q 值默认为 0。

经过足够多 episode 的训练,算法将收敛到最优值函数 $Q^*$,此时贪心策略 $\pi(s) = \arg\max_a Q(s,a)$ 即为最优策略。


% Method 2

\subsection{Q-Learning 原理与设计}

\subsubsection{基本原理}

Q-Learning 是一种时序差分(Temporal Difference, TD)学习算法,与 MC learning 的主要区别在于值函数的更新时机和方式。
MC learning 需要等待完整 episode 结束后才能计算回报 $G_t$ 并更新 Q 值,而 Q-Learning 在每一步转移后即可进行更新,无需等待终止状态。

Q-Learning 的核心更新规则基于贝尔曼最优方程,使用单步转移的即时奖励和下一状态的最优值估计来逼近真实 Q 值:
\begin{equation}
    \label{eq:q_learning_update}
    Q(s,a) \leftarrow Q(s,a) + \alpha \left[ r + \gamma \max_{a'} Q(s', a') - Q(s,a) \right]
\end{equation}
其中 $(s, a, r, s')$ 是观察到的一步转移。
与 MC learning 使用完整回报 $G_t$ 不同,Q-Learning 使用 TD 目标 $r + \gamma \max_{a'} Q(s', a')$ 作为 Q 值的估计。
这一目标结合了即时奖励 $r$ 和下一状态的最优值估计 $\max_{a'} Q(s', a')$,形成了自举(bootstrapping)式的更新。


\subsubsection{算法设计}

Q-Learning 的训练过程如算法 \ref{alg:q_learning} 所示。
相比 MC learning,该算法无需维护 episode 缓冲区,每步转移后立即更新 Q 值。

\begin{algorithm}[h]
    \caption{Q-Learning 算法}
    \small
    \label{alg:q_learning}
    \begin{algorithmic}[1]
        \Require 环境 $\text{env}$,最大训练 episode 数 $N$
        \Ensure 学习得到的 Q 值函数 $Q$
        
        \State 初始化:$Q(s,a) \leftarrow 0$ 对所有 $s \in \mathcal{S}, a \in \mathcal{A}$
        \State 设置超参数:$\alpha$, $\epsilon$, $\gamma$
        
        \For{$episode = 1$ to $N$}
            \State $s \leftarrow \text{env.reset()}$
            
            \While{episode 未结束}
                \State 以 $\epsilon$-Greedy 策略选择动作:
                \State \quad $a \leftarrow \begin{cases} 
                    \arg\max_{a'} Q(s, a') & \text{概率 } 1-\epsilon \\
                    \text{random}(\mathcal{A}) & \text{概率 } \epsilon
                \end{cases}$
                \State $s', r \leftarrow \text{env.step}(a)$
                \State $Q(s, a) \leftarrow Q(s, a) + \alpha \left[ r + \gamma \max_{a'} Q(s', a') - Q(s, a) \right]$ \Comment{TD 更新}
                \State $s \leftarrow s'$
            \EndWhile
        \EndFor
        
        \Return $Q$
    \end{algorithmic}
\end{algorithm}

算法在每次状态转移后(第 10 行)立即执行 TD 更新。
更新目标 $r + \gamma \max_{a'} Q(s', a')$ 结合了当前步的即时奖励和对未来回报的估计,
这种单步前瞻(one-step lookahead)的方式使得算法能够快速传播价值信息,加速学习过程。
Q 函数同样实现为字典结构,存储方式与 MC learning 完全相同。

% Method 3

\subsection{Approximate Q-Learning 原理与设计}

前述两种方法(MC learning 和 Q-Learning)在实现 Q 函数时,都是直接记录每个状态-动作对的值,
可以视作在填一张表格:当表格填满时,意味着所有状态-动作对都被访问过,从而完整得到 Q 函数。
智能体在决策时,通过查表得到当前状态下的最优 Q 值,进而做出最佳的动作。

然而,这同时也意味着如果某个状态-动作对从未被访问过,其 Q 值就为初始值,即使见过与其相似的状态-动作对,
智能体也无法做出合理的决策;
此外,如前所述,Pacman 游戏的状态空间随地图尺寸呈指数级增长,当地图稍大时,
Q 函数就变得难以学习。

Approximate Q-Learning 方法通过引入线性函数逼近和状态特征设计,解决了上述问题。

\subsubsection{线性函数逼近}

线性函数逼近(linear function approximation)通过参数化方式表示 Q 函数,将 Q 函数从离散的查表转换为连续函数逼近。
其核心思想是将 Q 值表示为特征的线性组合:
\begin{equation}
    \label{eq:approx_q}
    Q(s,a) = \sum_{i=1}^n w_i f_i(s,a) = \mathbf{w}^\top \mathbf{f}(s,a)
\end{equation}
其中 $\mathbf{f}(s,a) = [f_1(s,a), f_2(s,a), \ldots, f_n(s,a)]^\top$ 是从状态-动作对提取的特征向量,
$\mathbf{w} = [w_1, w_2, \ldots, w_n]^\top$ 是对应的权重向量。
特征函数 $f_i(s,a)$ 将高维状态空间映射到低维特征空间,捕获对决策有用的关键信息,
而权重 $w_i$ 则刻画了各特征对 Q 值的贡献程度。

在这一表示下,Q-Learning 的更新规则转化为对权重向量的更新。
给定一步转移 $(s, a, r, s')$,TD 误差(temporal difference error)定义为
\begin{equation}
    \label{eq:td_error}
    \delta = r + \gamma \max_{a'} Q(s', a') - Q(s,a)
\end{equation}
该误差衡量了当前 Q 值估计与 TD 目标之间的差距。
利用梯度下降法最小化 TD 误差的平方,可得权重更新规则:
\begin{equation}
    \label{eq:weight_update}
    \mathbf{w} \leftarrow \mathbf{w} + \alpha \cdot \delta \cdot \mathbf{f}(s,a)
\end{equation}
该更新规则沿着减小 TD 误差的方向调整权重,其中学习率 $\alpha$ 控制更新步长。

相比表格式方法,线性函数逼近具有显著优势。
首先,参数量大幅减少,从 $|\mathcal{S}| \times |\mathcal{A}|$ 降至特征维度 $n$(通常 $n \ll |\mathcal{S}| \times |\mathcal{A}|$)。
更重要的是,智能体通过线性函数逼近实现了泛化能力:
类似于插值,当两个不同的状态-动作对 $(s_1, a_1)$ 和 $(s_2, a_2)$ 提取出相似的特征,即 $\mathbf{f}(s_1, a_1) \approx \mathbf{f}(s_2, a_2)$ 时,
它们将得到相似的 Q 值估计 $\mathbf{w}^\top \mathbf{f}(s_1, a_1) \approx \mathbf{w}^\top \mathbf{f}(s_2, a_2)$。
这意味着智能体处理未见过的新状态时,可以利用隐含在权重中的知识做出合理的决策。

因此,通过线性逼近,可以对不完全匹配已知的 $(s,a)$ 估计出合理的值,从而不需要在训练中"完全覆盖表格"。
从另一个角度看,一次权重更新会影响所有具有相似特征的状态-动作对,也可以认为是学习效率提高了。

\subsubsection{特征设计}

% 特征设计是线性函数逼近方法的核心,直接决定了算法的性能上限。
% 特征应当捕获与决策相关的关键信息,同时保持维度适中以确保学习效率。

特征设计对于 Approximate Q-Learning 的实现也至关重要。
合理的特征能够有效地捕捉状态-动作对的信息,直接影响到 Q 值的估计质量和泛化能力。
同时,特征维度过高则会使智能体难以学习。
本实验设计的特征如表 \ref{tab:features} 所示。

\begin{table}[h]
    \renewcommand{\arraystretch}{1.5}
    \setlength{\tabcolsep}{8pt}
    \centering
    \caption{Approximate Q-Learning 特征设计}
    \small
    \begin{tabular}{lll}
        \toprule
        \textbf{特征名称} & \textbf{含义} & \textbf{取值范围} \\
        \midrule
        \texttt{bias} & 偏置项 & 1.0 \\
        \midrule
        \texttt{\#-of-normal-ghosts-1-step-away} & 1 步内的普通 ghost 数量 & 整数 \\
        \texttt{\#-of-scared-ghosts-1-step-away} & 1 步内的恐惧 ghost 数量 & 整数 \\
        \texttt{eats-scared-ghost} & 能否 1 步吃掉恐惧 ghost & $\{0, 1\}$ \\
        \texttt{closest-scared-ghost} & 最近恐惧 ghost 距离 & $[0, 1]$ \\
        \texttt{scared-timer} & 恐惧状态剩余时间 & $[0, 1]$ \\
        \midrule
        \texttt{eats-capsule} & 能否 1 步吃到胶囊 & $\{0, 1\}$ \\
        \texttt{closest-capsule} & 最近胶囊距离 & $[0, 1]$ \\
        \midrule
        \texttt{eats-food} & 能否 1 步吃到食物 & $\{0, 1\}$ \\
        \texttt{closest-food} & 最近食物距离 & $[0, 1]$ \\
        \bottomrule
    \end{tabular}
    \label{tab:features}
\end{table}

其中,偏置项用于捕捉状态的基本特征,使得 Q 值的初始估计不为零。
距离类特征(\texttt{closest-*})通过除以地图宽度与高度的乘积归一化到 $[0,1]$ 区间,
\texttt{scared-timer} 通过除以恐惧状态持续时间(40 个时间步)归一化。
这些特征涵盖了 Pacman 游戏中与决策相关的主要信息:
食物和胶囊的位置与可达性、普通 ghost 的威胁、恐惧 ghost 的捕获机会、以及恐惧状态的时间窗口。

\subsubsection{算法设计}

Approximate Q-Learning 的训练过程如算法 \ref{alg:approx_q_learning} 所示。

\begin{algorithm}[h]
    \caption{Approximate Q-Learning 算法}
    \small
    \label{alg:approx_q_learning}
    \begin{algorithmic}[1]
        \Require 环境 $\text{env}$,特征提取器 $\phi$,最大训练 episode 数 $N$
        \Ensure 学习得到的权重向量 $\mathbf{w}$
        
        \State 初始化:$\mathbf{w} \leftarrow \mathbf{0}$(所有权重初始化为 0)
        \State 设置超参数:$\alpha$, $\epsilon$, $\gamma$
        
        \For{$episode = 1$ to $N$}
            \State $s \leftarrow \text{env.reset()}$
            
            \While{episode 未结束}
                \State 以 $\epsilon$-Greedy 策略选择动作:
                \State \quad $a \leftarrow \begin{cases} 
                    \arg\max_{a'} \mathbf{w}^\top \phi(s, a') & \text{概率 } 1-\epsilon \\
                    \text{random}(\mathcal{A}) & \text{概率 } \epsilon
                \end{cases}$
                \State $s', r \leftarrow \text{env.step}(a)$
                \State $\delta \leftarrow r + \gamma \max_{a'} \mathbf{w}^\top \phi(s', a') - \mathbf{w}^\top \phi(s, a)$ \Comment{计算 TD 误差}
                \State $\mathbf{w} \leftarrow \mathbf{w} + \alpha \cdot \delta \cdot \phi(s, a)$ \Comment{更新权重}
                \State $s \leftarrow s'$
            \EndWhile
        \EndFor
        
        \Return $\mathbf{w}$
    \end{algorithmic}
\end{algorithm}

算法的核心在于第 10-11 行的权重更新机制。
首先计算 TD 误差 $\delta$(第 10 行),其衡量了当前估计 $\mathbf{w}^\top \phi(s, a)$ 与 TD 目标 $r + \gamma \max_{a'} \mathbf{w}^\top \phi(s', a')$ 之间的差距。
随后沿着特征向量 $\phi(s, a)$ 的方向更新权重(第 11 行),更新幅度由 TD 误差 $\delta$ 和学习率 $\alpha$ 共同决定。

直观地理解,若某状态-动作对的实际回报高于当前估计($\delta > 0$),则增大与该对相关的特征权重;
反之若实际回报低于估计($\delta < 0$),则减小相应权重。
通过这种方式,算法逐步调整权重向量,使得 Q 值估计 $\mathbf{w}^\top \phi(s,a)$ 逼近真实的最优 Q 函数。

% 相比 Q-Learning,Approximate Q-Learning 在动作选择时(第 7 行)需要遍历所有可能动作并计算 $\mathbf{w}^\top \phi(s, a')$,
% 这一计算复杂度为 $O(|\mathcal{A}| \cdot n)$,其中 $n$ 是特征维度。
% 由于 Pacman 的动作空间很小($|\mathcal{A}| = 5$)且特征维度适中($n \approx 10$),这一开销完全可接受。

% 经过充分训练,权重向量 $\mathbf{w}$ 收敛后,贪心策略 $\pi(s) = \arg\max_a \mathbf{w}^\top \phi(s,a)$ 即可在未见过的状态上进行有效的泛化决策。

通过特征提取,将大量原始状态映射到少量特征组合,减小了“表格”的大小,从而易于学习;
再结合线性逼近带来的泛化性能,Approximate Q-Learning 能够在大规模状态空间中有效完成学习。

