\section{概述}
\label{sec:intro}

\subsection{问题背景}

代码类型推断是程序分析领域的重要问题。
对于动态类型语言(如Python、JavaScript),类型信息的缺失导致编译时无法检测类型错误,
降低了代码的可维护性和可靠性。
Python 3.5引入的类型提示机制~\cite{pep484}允许开发者为代码添加类型标注,
但为遗留项目补充类型标注仍需大量人工工作。

近年来,基于深度学习的类型推断方法逐渐兴起。
与传统基于规则的方法相比,深度学习方法能够从大量代码数据中学习类型模式。
图神经网络(Graph Neural Networks, GNN)为这一任务提供了新的技术路径:
程序代码可以表示为图结构(如抽象语法树、数据流图),
图神经网络通过消息传递机制能够捕获代码的结构特征和语义依赖。

\subsection{实验任务}

在《图神经网络导论》课程中,系统介绍了图论基础、图嵌入、图神经网络原理、鲁棒性与可扩展性,
以及在生物信息、自然语言处理和代码智能中的应用。
其中课程第10章"代码智能中的图神经网络"介绍了代码的图表示方法(AST、CFG、DFG、CPG等)
和基于GNN的代码分析任务(变量误用检测、类型推断、漏洞检测等)。

课程实验提供了两个可选题目:代码漏洞检测和代码类型推断。
这里选择代码类型推断任务,复现Typilus方法并进行了一些探索。
Typilus\cite{typilus2020}
将Python代码表示为图结构,使用GGNN进行类型预测。

\subsection{实验内容}

本实验报告的主要内容包括:

(1)复现Typilus模型。
在NaturalCC框架~\cite{naturalcc2022}下实现Typilus方法,
包括代码属性图的构建、GGNN编码器的训练和类型预测流程。
在实验过程中解决了依赖版本冲突、词典格式不兼容、显存溢出等多个技术问题。

(2)设计对比实验。
实现基于Transformer的序列模型作为baseline,
在相同数据集下与Typilus进行对比。
设计7组消融实验(模型架构、嵌入维度、网络层数、学习率、Dropout、批量大小等),
分析不同超参数配置对模型性能的影响。

(3)开发实验工具。
实现数据格式转换工具(Typilus图格式 → Transformer序列格式),
开发实验配置生成、批量运行和结果分析脚本,提升实验效率。

\subsection{报告组织}

本报告共8章:
第2章综述图神经网络和代码智能相关工作(对应课程内容);
第3章介绍Typilus方法;
第4章描述数据准备和环境配置;
第5章展示Typilus复现过程和结果;
第6章分析Transformer对比实验;
第7章总结实验工作;
第8章分享学习收获。
