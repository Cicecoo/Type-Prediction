\section{概述}
\label{sec:intro}

\subsection{问题背景}

代码类型推断是程序分析领域的重要问题。
对于动态类型语言(如Python、JavaScript),类型信息的缺失导致编译时无法检测类型错误,
降低了代码的可维护性和可靠性。
Python 3.5引入的类型提示机制\cite{pep484}允许开发者为代码添加类型标注,
但GitHub上大量Python项目仍缺乏完整的类型标注\cite{typilus2020},
手动补充类型信息需要大量人工工作。

近年来,基于深度学习的类型推断方法逐渐兴起。
与传统基于规则的方法相比,深度学习方法能够从大量代码数据中学习类型模式。
图神经网络(Graph Neural Networks, GNN)为这一任务提供了新的技术路径:
程序代码可以表示为图结构(如抽象语法树、数据流图),
图神经网络通过消息传递机制能够捕获代码的结构特征和语义依赖。

\subsection{实验任务}

在《图神经网络导论》课程中,系统介绍了图论基础、图嵌入、图神经网络原理、鲁棒性与可扩展性,
以及在生物信息、自然语言处理和代码智能中的应用。
其中课程第10章"代码智能中的图神经网络"介绍了代码的图表示方法(AST、CFG、DFG、CPG等)
和基于GNN的代码分析任务(变量误用检测、类型推断、漏洞检测等)。

课程实验提供了两个可选题目:代码漏洞检测和代码类型推断。
这里选择代码类型推断任务,在naturalcc\cite{naturalcc2020}框架下复现了Typilus\cite{typilus2020}方法,并进行了一些探索;
进一步基于transformer模型实现了序列化的类型预测方法作为对比。

\subsection{实验内容}

本实验的主要内容包括:

首先,搭建Typilus实验环境,包括数据集爬取和NaturalCC框架配置。
通过修改数据处理脚本,摆脱了Docker容器的依赖,直接在本地环境完成数据准备流程。
在复现过程中解决了依赖版本冲突、词典格式不兼容、显存溢出(OOM)等技术问题。
此外,针对验证过程中距离矩阵计算导致的显存爆炸问题,采用分块计算策略,避免显式构造完整矩阵。
最终成功运行Typilus训练流程。

进一步,使用不同的配置训练,
探索学习率、模型尺寸(嵌入维度、网络层数)、Dropout、批量大小等超参数对性能的影响。

此外,实现基于Transformer的序列模型作为对比baseline,
验证图结构建模相比序列建模的作用。
过程中开发了数据格式转换工具和实验管理脚本。

\subsection{报告组织}

本报告后续内容安排如下:
第2章综述图神经网络和代码智能的相关工作,对应课程所学内容;
第3章介绍Typilus方法的设计思路和模型架构;
第4章描述数据准备和环境配置过程;
第5章展示Typilus的复现过程和结果;
第6章分析Transformer对比实验的设计和发现;
第7章总结实验工作并讨论不足;
第8章分享学习收获和思考。
